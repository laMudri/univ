\documentclass{article}

\usepackage{amsmath}
\usepackage{color}
\usepackage[a4paper,margin=1in]{geometry}
\usepackage{graphicx}
\usepackage[utf8]{inputenc}
\usepackage{listings}

\lstset
{language=Haskell
,basicstyle=\footnotesize
,extendedchars=true
,literate={é}{{\'e}}1
,frame=single
}

\begin{document}
\title{Computer Graphics and Image Processing -- supervision 2}
\author{James Wood}
\maketitle

\begin{enumerate}
  \item Using matrices allows us to compose transformations quickly, via matrix multiplication. This can be used to optimize the application of a set sequence of transformations.
    Homogeneous co\"ordinates are used to allow for more transformations than can be achieved with a simple Cartesian co\"ordinate system. Notably, translation can be encoded as an $n+1$-dimensional shear.
  \item See scanned page.
  \item
    Suppose that the B\'ezier curves to be joined have functions $P$ and $Q$. Furthermore, \[
      P(t) = (1-t)^3 \cdot P_0 + 3 \cdot t \cdot (1-t)^2 \cdot P_1 + 3 \cdot t^2 \cdot (1-t) \cdot P_2 + t^3 \cdot P_3
    \], and $Q$ is defined similarly. We are trying to achieve continuity between $P$ at $1$ and $Q$ at $0$.
    \begin{enumerate}
      \item $C_0$ continuity requires only that the curves join to one-another, i.e, $P(1) = Q(0)$. $P(1) = P_3$ and $Q(0) = Q_0$, so we require only
        \begin{equation}
          P_3 = Q_0. \tag{PQ0}
        \end{equation}
      \item
        $C_1$ continuity further requires that the derivatives of $P$ and $Q$ are equal where they are supposed to meet, i.e, $P'(1) = Q'(0)$. We have
        \begin{align*}
          P'(t) =& {} - 3 \cdot (1-t)^2 \cdot P_0
          \\
          & {} + 3 \cdot ((1-t)^2 - 2 \cdot t \cdot (1-t)) \cdot P_1
          \\
          & {} + 3 \cdot (2 \cdot t \cdot (1-t) - t^2) \cdot P_2
          \\
          & {} + 3 \cdot t^2 \cdot P_3
        \end{align*}
        or equivalently
        \begin{align*}
          P'(t) = 3 \cdot (& {} - (1-t)^2 \cdot P_0
          \\
          & {} + ((1-t)^2 - 2 \cdot t \cdot (1-t)) \cdot P_1
          \\
          & {} + (2 \cdot t \cdot (1-t) - t^2) \cdot P_2
          \\
          & {} + t^2 \cdot P_3).
        \end{align*}
        Similar holds for $Q$. The new requirement simplifies as follows:
        \begin{align*}
          P'(1) &= Q'(0)
          \\
          3 \cdot (-P_2 + P_3) &= 3 \cdot (-Q_0 + Q_1)
          \\
          P_3 - P_2 &= Q_1 - Q_0 \tag{PQ1}
        \end{align*}
      \item
        $C_2$ continuity requires also that $P''(1) = Q''(0)$. Building upon the previous working, we have
        \begin{align*}
          P'(t) = 3 \cdot (& \phantom{{}+{}} 2 \cdot (1-t) \cdot P_0
          \\
          & {} + (-2 \cdot (1-t) - 2 \cdot ((1-t) - t)) \cdot P_1
          \\
          & {} + (2 \cdot ((1-t) - t) - 2 \cdot t) \cdot P_2
          \\
          & {} + 2 \cdot t \cdot P_3)
        \end{align*}
        or equivalently
        \begin{align*}
          P'(t) = 6 \cdot (& \phantom{{}+{}} (1-t) \cdot P_0
          \\
          & {} + (-2 \cdot (1-t) + t) \cdot P_1
          \\
          & {} + ((1-t) - 2 \cdot t) \cdot P_2
          \\
          & {} + t \cdot P_3).
        \end{align*}
        Again, we simplify the new requirement:
        \begin{align*}
          P''(1) &= Q''(0)
          \\
          6 \cdot (P_1 - 2 \cdot P_2 + P_3) &= 6 \cdot (Q_0 - 2 \cdot Q_1 + Q_2)
          \\
          P_1 - 2 \cdot P_2 + P_3 &= Q_0 - 2 \cdot Q_1 + Q_2 \tag{PQ2}
        \end{align*}
        If we introduce a third curve, $R$, we will get the following requirements for it to have $C_2$ continuity where it meets $Q$:
        \begin{align*}
          Q_3 &= R_0 \tag{QR0}
          \\
          Q_3 - Q_2 &= R_1 - R_0 \tag{QR1}
          \\
          Q_1 - 2 \cdot Q_2 + Q_3 &= R_0 - 2 \cdot R_1 + R_2 \tag{QR2}
        \end{align*}
        Both PQ2 and QR2 put restrictions on $Q_1$ and $Q_2$. This means that fixing the join between $P$ and $Q$ requires us to consider the join between $Q$ and $R$.
    \end{enumerate}
    \par
  \item
    \lstinputlisting{S2.hs}
    \includegraphics{S2.png}
    \centering
\end{enumerate}

\end{document}

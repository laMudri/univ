\documentclass{article}

\usepackage{amsmath}
\usepackage[margin=1in]{geometry}

\newcommand{\parto}{\rightharpoonup}

\begin{document}
\title{An alternative answer to exercise 10.b}
\author{James Wood}
\maketitle

In this piece, I aim to prove that the set answer to exercise 10.b of the Computation Theory exercise sheet, namely, defining $g_{n+1} := \rho^0(g_n \circ \mathtt{succ} \circ \mathtt{zero}^0, g_n \circ \mathtt{proj}_2^2)$, is in fact equivalent to the answer I gave in the supervision, using the definition $g_{n+1} := g_n \circ \rho^0(\mathtt{succ} \circ \mathtt{zero}^0, g_n \circ \mathtt{proj}_2^2)$. I do this by proving the following equality for all (primitive recursive) $f$, $c$, and $x$.
\begin{align}
  \rho^0(f \circ c, f \circ \mathtt{proj}_2^2)(x) = (f \circ \rho^0(c, f \circ \mathtt{proj}_2^2))(x)
\end{align}
By associativity of composition and assumption of extensional equality on functions, it should be clear that the desired result is a corollary.

To prove (1), we proceed by induction on $x$. The base case follows by calculation. Explicitly,
\begin{align*}
  \rho^0(f \circ c, f \circ \mathtt{proj}_2^2)(0)
  & = (f \circ c)()
  \\
  & = f(c())
\end{align*}
and
\begin{align*}
  (f \circ \rho^0(c, f \circ \mathtt{proj}_2^2))(0)
  & = f(\rho^0(c, f \circ \mathtt{proj}_2^2)(0))
  \\
  & = f(c())
\end{align*}

Calculation also gets us a long way in the step case. Note:
\begin{align*}
  \rho^0(f \circ c, f \circ \mathtt{proj}_2^2)(x+1)
  & = (f \circ \mathtt{proj}_2^2)(x, \rho^0(f \circ c, f \circ \mathtt{proj}_2^2)(x))
  \\
  & = f(\mathtt{proj}_2^2(x, \rho^0(f \circ c, f \circ \mathtt{proj}_2^2)(x)))
  \\
  & = f(\rho^0(f \circ c, f \circ \mathtt{proj}_2^2)(x))
\end{align*}
and
\begin{align*}
  (f \circ \rho^0(c, f \circ \mathtt{proj}_2^2))(x+1)
  & = f(\rho^0(c, f \circ \mathtt{proj}_2^2)(x+1))
  \\
  & = f((f \circ \mathtt{proj}_2^2)(x, \rho^0(c, f \circ \mathtt{proj}_2^2)(x)))
  \\
  & = f(f(\mathtt{proj}_2^2(x, \rho^0(c, f \circ \mathtt{proj}_2^2)(x))))
  \\
  & = f(f(\rho^0(c, f \circ \mathtt{proj}_2^2)(x)))
\end{align*}
The inductive hypothesis then tells us
\begin{align*}
  \rho^0(f \circ c, f \circ \mathtt{proj}_2^2)(x) = f(\rho^0(c, f \circ \mathtt{proj}_2^2)(x))
\end{align*}
and hence
\begin{align*}
  f(\rho^0(f \circ c, f \circ \mathtt{proj}_2^2)(x)) = f(f(\rho^0(c, f \circ \mathtt{proj}_2^2)(x))),
\end{align*}
as required.

\end{document}

\documentclass{article}

\usepackage[margin=1in]{geometry}
\usepackage{listings}

\begin{document}
\title{C/C++ -- supervision 2}
\author{James Wood}
\maketitle

\section{Lecture 4}
\begin{enumerate}
  \item
    \begin{enumerate}
      \item \texttt{i = 1}.
      \item \texttt{i} is compiler-defined. Usually, a value of 4 or 8 is used.
      \item As before, 4 or 8.
      \item \texttt{i = 5} because b is declared as an array containing 5 items of size 1.
      \item \texttt{i} is the size of pointers, as defined by the compiler (usually based on the size of addresses on the machine). This is 4 for a 32-bit compiler and 8 for a 64-bit compiler.
      \item \texttt{i >= sizeof(int) + sizeof(char)}, but ultimately determined by the compiler. Often, structs are padded out so that they align well when put in an array.
      \item \texttt{i} is the size of a pointer, since the size of array parameters is ignored and they are just considered to be pointers.
      \item Similarly, \texttt{i} is the size of a pointer.
    \end{enumerate}
  \item \lstinputlisting[language=C]{heap.c}
  \item A complete heap can be implemented as an array, with the head being at position $0$ and node $n$'s immediate children being at positions $2 \cdot n + 1$ and $2 \cdot n + 2$. Then, we can use an algorithm that re\"arranges the given array into a heap in $O~(n)$ time.
\end{enumerate}

\section{Lecture 5}
\begin{enumerate}
  \item \lstinputlisting[language=C++]{linkList.cpp}
\end{enumerate}

\section{2013 -- Paper 3 -- Question 3}
\begin{enumerate}
  \item
    \begin{enumerate}
      \item On the legacy processor, the struct is being stored packed, meaning that it only takes up the mininum of 5 bytes (4 for \texttt{val}, 1 for \texttt{flags}). On the other systems, the struct gets padded. The first system has 32-bit longs, and 4-byte alignment. The second system has 64-bit longs with the same padding. The third system has 64-bit longs with 8-byte alignment. \texttt{sizeof(struct Elem)} is 5, 8, 12, and 16, respectively.
      \item Firstly, we have to assume that the legacy processor stores signed longs in big-endian order. Then, we can do the following:
        \begin{lstlisting}[language=C]
char u[5 * NITEMS];
fread(file, 1, sizeof(u), u);
size_t i;
struct Elem v[NITEMS];
for (i = 0; i < NITEMS; i++) {
  v[i].val = (signed long)u[0] << 24 + (signed long)u[1] << 16 +
             (signed long)u[2] <<  8 + (signed long)u[3];
  v[i].flags = u[4];
  u += 5;
}
        \end{lstlisting}
    \end{enumerate}
\end{enumerate}

\section{2010 -- Paper 3 -- Question 6}
\begin{enumerate}
  \item \lstinputlisting[language=C]{dll.c}
  \item
\end{enumerate}

\end{document}

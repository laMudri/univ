\message{ !name(de.tex)}\documentclass[11pt]{article}
%Gummi|063|=)
\title{\textbf{Differential Equations}}
\author{James Wood}
\date{2014-10-09}
\usepackage{hyperref}
\usepackage{amsfonts}
\usepackage{amsmath}
\usepackage{amssymb}
\usepackage{mathtools}
\newcommand*\C{\ensuremath{\mathbb C}}
\newcommand*\R{\ensuremath{\mathbb R}}
\newcommand*\id{\iota}
\newcommand*\cd{\cdot}
\newcommand*\prg{\paragraph}
\newcommand*\pt{\prescript}
\newcommand*\conj[1]{\overline{#1}}
\DeclareMathOperator\abs{abs}
\DeclareMathOperator\all{all}
\DeclareMathOperator\any{any}
\DeclareMathOperator\card{card}
\DeclareMathOperator\neg{neg}

\begin{document}

\message{ !name(de.tex) !offset(190) }

\prg{}Finally, we cover the notion of indefinite integration. This can be expressed in terms of definite integration as follows:
\[
\int f=\int^\C_\id f
\]
\prg{}Here, $\C$ gives us (but does not map directly to) the constant of integration.

\end{document}

\message{ !name(de.tex) !offset(-36) }

\message{ !name(vm.tex)}\documentclass[11pt]{article}
%Gummi|063|=)
\title{\textbf{Vectors and matrices}}
\author{James Wood}
\date{2014-10-11}
\usepackage{hyperref}
\usepackage{amsfonts}
\usepackage{amsmath}
\usepackage{amssymb}
\usepackage{mathtools}
\usepackage{esvect}
\newcommand*\C{\ensuremath{\mathbb C}}
\newcommand*\R{\ensuremath{\mathbb R}}
\newcommand*\id{\iota}
\newcommand*\cd{\cdot}
\newcommand*\prg{\paragraph}
\newcommand*\pt{\prescript}
\newcommand*\conj[1]{\overline{#1}}
\DeclareMathOperator{\abs}{abs}
\DeclareMathOperator{\all}{all}
\DeclareMathOperator{\any}{any}
\DeclareMathOperator{\card}{card}
\DeclareMathOperator{\mgn}{mgn}
\DeclareMathOperator{\prj}{prj}

\begin{document}

\message{ !name(vm.tex) !offset(326) }


\subsection{Spanning sets and bases}
\prg{}A set of vectors is said to span a space if every point in the space can be translated to any other point in the space using a linear combination of vectors from the set. A basis is a spanning set such that if any element were to be removed from that set, it would no longer be a spanning set.

\prg{}An example of a basis is $\{\mathbf{\hat i},\mathbf{\hat j}\}$ in 2D Euclidean space. If the vector $\mathbf{\hat i}+\mathbf{\hat j}$ is added to this set, it will still be a spanning set, but not a basis.

\end{document}

\message{ !name(vm.tex) !offset(-38) }

\documentclass[11pt]{article}
%Gummi|063|=)
\title{\textbf{Vectors and matrices}}
\author{James Wood}
\date{2014-10-11}
\usepackage{hyperref}
\usepackage{amsfonts}
\usepackage{amsmath}
\usepackage{amssymb}
\usepackage{mathtools}
\newcommand*\C{\ensuremath{\mathbb C}}
\newcommand*\R{\ensuremath{\mathbb R}}
\newcommand*\id{\iota}
\newcommand*\cd{\cdot}
\newcommand*\prg{\paragraph}
\newcommand*\pt{\prescript}
\newcommand*\conj[1]{\overline{#1}}
\DeclareMathOperator{\abs}{abs}
\DeclareMathOperator{\all}{all}
\DeclareMathOperator{\any}{any}
\DeclareMathOperator{\card}{card}

\begin{document}

\maketitle

\section*{Notations and conventions}
\prg{}The notation used here is under development. Details can be found at \url{http://fancyfahu.blogspot.co.uk}. Particularly, the arguments of subtraction and division are swapped.

\section{Complex numbers}
\subsection{Basic properties}
\prg{}If $z$ is a complex number, it can be expressed in the form $a+i\cd b$, where $a,b\in\mathbb R$ and $\pt 2 \cd i=1^-$. Complex numbers are closed under addition, subtraction, multiplication, division and exponentiation.

\subsection{Argand diagram}
Complex numbers hold the same information as real 2-vectors ($\R\times\R$), and hence complex numbers can be represented as 2-vectors visually. Also, addition of complex numbers corresponds to addition of 2-vectors. From this representation, the triangle inequality is obvious.
$$
\abs(z_0+z_1)\leq\abs z_0+\abs z_1
$$
The fact that the triangle inequality holds is the characteristic of complex numbers that makes them form a metric space.

\prg{}The following inequality also holds, and is identical:
$$
\abs(z_0-z_1)\geq\abs(\abs z_0-\abs z_1)
$$
\prg{}This is proved as follows. $\abs z_0=\abs(z_1+z_1-z_0)$. By the triangle inequality,
$\abs(z_1+(z_1-z_0))\leq\abs z_1+\abs(z_1-z_0)$. Hence, $\abs z_1-\abs z_0\leq\abs(z_1-z_0)$. A similar argument gives $\abs z_1\leq\abs z_0+\abs(z_0-z_1)$, and hence $\abs z_0-\abs z_1\leq\abs(z_0-z_1)$. $\abs(z_1-z_0)=\abs(z_0-z_1)$, so
$\abs(z_0-z_1)^-\leq\abs z_0-\abs z_1;\leq\abs(z_0-z_1)$. In other words, $\abs(\abs z_0-\abs z_1)\leq\abs(z_0-z_1)$, QED.

\subsection{Complex exponential}
The Taylor series of the $\exp:\R\rightarrow\R$ function can be used to extend its domain (and, incidentally, codomain).
\[
\begin{aligned}
\exp&=1+\id+\frac{2!}{\pt 2\cd\id}+\hdots; \\
    &=\sum^0_\infty z:\frac{\id!}{\pt\id\cd z}
\end{aligned}
\]
\prg{}This funciton converges for all $\C$.

\subsubsection{Multiplication}
\prg{}Here, we prove that $\all(\exp z_0\cd\exp z_1=\exp(z_0+z_1))\,(\C\times\C)$. Let $f\,z\,n=\frac{n!}{\pt n\cd z}$
\[
\begin{aligned}
\exp z_0\cd\exp z_1&=\left(\sum^0_\infty z_0,z_1:f\,z_0\right)\cd\left(\sum^0_\infty z_0,z_1:f\,z_1\right); \\
                  &=\sum^0_\infty z_0,z_1:\left(\sum^0_\infty\kappa:f\,z_0\,\id\cd f\,z_1\,\kappa\right)\id; \\
                  &=\sum^0_\infty z_0,z_1:\left(\sum_{\leq|0,\id|\geq}\kappa:f\,z_0\,(\id-\kappa)\cd f\,z_1\,\id\right)\id;\quad\textrm{taking sums along minor diagonals}\\
                  &=\sum^0_\infty z_0,z_1:\left(\sum_{\leq|0,\id|\geq}\kappa:\frac{(\id-\kappa)!}{\pt{\id-\kappa}{\cd}z_0}\cd \frac{\id!}{\pt\id\cd z_1}\right)\id; \\
                  &=\sum^0_\infty z_0,z_1:\frac{\id!}{1}\cd\left(\sum_{\leq|0,\id|\geq}\kappa:\frac{(\id-\kappa)!\cd\id!}{\kappa!}\cd\pt{\id-\kappa}{\cd}z_0\cd\pt\id\cd z_1\right)\id; \\
                  &=\sum^0_\infty z_0,z_1:\frac{\id!}{\pt\id\cd(z_0+z_1)};\quad\textrm{reversing the binomial expansion} \\
                  &=\exp(z_0+z_1)
\end{aligned}
\]

\section{The complex exponential and Argand diagram}
\subsection{Trigonometric functions}
\prg{}The basic trigonometric functions can be defined using the complex exponential function and splitting the result into real and imaginary parts. On the Argand diagram (for real argument $\theta$), this is represented by plotting the point $\exp(i\cd\theta)$ and measuring the $x$ (for $\cos$) and $y$ (for $\sin$) parts of its co\"ordinates.
\[
\begin{aligned}
\exp(i\cd z)&=\sum^0_\infty\frac{\id!}{\pt\id\cd i\cd\pt\id\cd z} \\
&=\sum^0_\infty\frac{(2\cd\id)!}{\pt{2\cd\id}{\cd}i\cd\pt{2\cd\id}{\cd}z}+
  \sum^0_\infty\frac{(1+2\cd\id)!}{\pt{1+2\cd\id}{\cd}i\cd\pt{1+2\cd\id}{\cd}z} \\
&=\sum^0_\infty\frac{(2\cd\id)!}{\pt\id\cd(1^-)\cd\pt{2\cd\id}{\cd}z}+i\cd
  \sum^0_\infty\frac{(1+2\cd\id)!}{\pt\id\cd(1^-)\cd\pt{1+2\cd\id}{\cd}z} \\
&=\cos z+i\cd\sin z
\end{aligned}
\]

\subsection{Roots of unity}
\prg{}The equation $\pt n\cd\id=1$ (with $n\in\mathbb N$), being a polynomial equation of order $n$, has $n$ values that satisfy it. The obvious root is $1$, but there are others.
\[
\begin{aligned}
&\textrm{Let }\exp(\tau\cd i\cd k)=1;=\pt n\cd z \\
&\exp\frac{n}{\tau\cd i\cd k}=z \\
&z\in\exp\frac{n}{\tau\cd i\cd\mathbb Z} \\
\end{aligned}
\]
\prg{}Though $\mathbb Z$ is infinite, $\exp(\frac{n}{\tau\cd i}\cd\mathbb Z)=\exp(\frac{n}{\tau\cd i}\cd(n+\mathbb Z))$, due to the latter being a complete rotation around the Argand diagram of the former.

\prg{}The roots of unity for a given $n$ form a regular polygon about $0$ when adjacent roots are connected on the Argand diagram. This is related to the fact that the sum of all roots (for $n>1$) is equal to $0$:
\[
\begin{aligned}
&\textrm{Let }\pt{\leq|0,n|>}{\cd}\omega\textrm{ be the solutions to }1-\pt n\cd\id=0 \\
&1-\pt n\cd\omega=0 \quad\textrm{by definition of }\omega \\
&1-\pt n\cd\omega=(1-\omega)\cd\sum^0_n(\pt\id\cd\omega) \\
&0=(1-\omega)\cd\sum^0_n(\pt\id\cd\omega) \\
&\omega\neq 1\quad\therefore\quad\sum^0_n(\pt\id\cd\omega)=0
\end{aligned}
\]

\subsection{Complex logarithm}
\prg{}The $\ln$ function is defined as the inverse of the $\exp$ function:
\[
\begin{aligned}
\id&=r\cd\exp(i\cd\theta) \\
\ln&=\ln r+\ln(\exp(i\cd\theta)) \quad r\in\R\textrm{, so we can calculate }\ln r \\
\ln&=\ln r+i\cd\theta \quad\textrm{by definition of }\ln \\
\ln&=\ln\abs+i\cd\arg \quad\textrm{explicitly}
\end{aligned}
\]
\prg{}Since, for complex input, there are infinite classes of values which can be passed to $\exp$ to give the same output, the range of $\ln$ has to be restricted. Given the above expression, the obvious way is to restrict the range of $\arg$ to a $\tau$-sized half-open interval.

\subsection{Complex power function}
\prg{}The complex power function is derived as follows:
\[
\begin{aligned}
\pt\alpha\cd z&=\exp(\ln(\pt\alpha\cd z)) \\
&=\exp(\alpha\cd\ln z)
\end{aligned}
\]
\prg{}With the multivalued (list-producing) $\ln$ function, we note the following:
\[
\alpha\in\mathbb Q\iff\card\pt\alpha\cd z<\infty
\]

\subsection{Lines and circles in $\C$}
\subsubsection{Lines}
\prg{}Lines can be expressed in a parametric way, as they would be with vectors.
\[
\id=z+\lambda\cd w\textrm{, }\lambda\in\R
\]
\prg{}Compare to $\id=\mathbf a+\lambda\cd\mathbf d$. However, with complex numbers, the $\lambda$ can be removed by the following process:
\[
\begin{aligned}
\lambda&=\frac{w}{z-\id} \\
\conj\lambda&=\conj{\left(\frac{w}{z-\id}\right)} \\
\lambda&=\frac{\conj w}{\conj z-\conj\id} \quad\textrm{N.B: }\lambda\in\R \\
\frac{w}{z-\id}&=\frac{\conj w}{\conj z-\conj\id} \\
\conj w\cd(z-\id)&=w\cd(\conj z-\conj\id)
\end{aligned}
\]

\subsubsection{Circles}
\prg{}Circles are specified with a centre and a radius. By definition, every distance from a point on the circle to the centre is the radius.
\[
\abs(c-\id)=\rho
\]
\prg{}This can be expressed in an alternative form:
\[
\begin{aligned}
\pt 2\cd\abs(c-\id)&=\pt 2\cd\rho \quad\textrm{N.B: both $\abs(c-\id)$ and $\rho$ are positive.} \\
(c-\id)\cd(\conj c-\conj\id)&=\pt 2\cd\rho \\
c\cd\conj\id-\id\cd\conj c-\id\cd\conj\id&=c\cd\conj c-\pt 2\cd\rho
\end{aligned}
\]

\end{document}

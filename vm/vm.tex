\documentclass[11pt]{article}
%Gummi|063|=)
\title{\textbf{Vectors and matrices}}
\author{James Wood}
\date{2014-10-11}
\usepackage{hyperref}
\usepackage{amsfonts}
\usepackage{amsmath}
\usepackage{amssymb}
\usepackage{mathtools}
\usepackage{esvect}
\newcommand*\C{\ensuremath{\mathbb C}}
\newcommand*\R{\ensuremath{\mathbb R}}
\newcommand*\id{\iota}
\newcommand*\cd{\cdot}
\newcommand*\prg{\paragraph}
\newcommand*\pt{\prescript}
\newcommand*\conj[1]{\overline{#1}}
\DeclareMathOperator{\abs}{abs}
\DeclareMathOperator{\all}{all}
\DeclareMathOperator{\any}{any}
\DeclareMathOperator{\card}{card}
\DeclareMathOperator{\mgn}{mgn}
\DeclareMathOperator{\prj}{prj}

\begin{document}

\maketitle

\section*{Notations and conventions}
\prg{}The notation used here is under development. Details can be found at \url{http://fancyfahu.blogspot.co.uk}. Particularly, the arguments of subtraction and division are swapped.

\section{Complex numbers}
\subsection{Basic properties}
\prg{}If $z$ is a complex number, it can be expressed in the form $a+i\cd b$, where $a,b\in\mathbb R$ and $\pt 2 \cd i=1^-$. Complex numbers are closed under addition, subtraction, multiplication, division and exponentiation.

\subsection{Argand diagram}
Complex numbers hold the same information as real 2-vectors ($\R\times\R$), and hence complex numbers can be represented as 2-vectors visually. Also, addition of complex numbers corresponds to addition of 2-vectors. From this representation, the triangle inequality is obvious.
$$
\abs(z_0+z_1)\leq\abs z_0+\abs z_1
$$
The fact that the triangle inequality holds is the characteristic of complex numbers that makes them form a metric space.

\prg{}The following inequality also holds, and is identical:
$$
\abs(z_0-z_1)\geq\abs(\abs z_0-\abs z_1)
$$
\prg{}This is proved as follows. $\abs z_0=\abs(z_1+z_1-z_0)$. By the triangle inequality,
$\abs(z_1+(z_1-z_0))\leq\abs z_1+\abs(z_1-z_0)$. Hence, $\abs z_1-\abs z_0\leq\abs(z_1-z_0)$. A similar argument gives $\abs z_1\leq\abs z_0+\abs(z_0-z_1)$, and hence $\abs z_0-\abs z_1\leq\abs(z_0-z_1)$. $\abs(z_1-z_0)=\abs(z_0-z_1)$, so
$\abs(z_0-z_1)^-\leq\abs z_0-\abs z_1;\leq\abs(z_0-z_1)$. In other words, $\abs(\abs z_0-\abs z_1)\leq\abs(z_0-z_1)$, QED.

\subsection{Complex exponential}
The Taylor series of the $\exp:\R\rightarrow\R$ function can be used to extend its domain (and, incidentally, codomain).
\[
\begin{aligned}
\exp&=1+\id+\frac{2!}{\pt 2\cd\id}+\hdots; \\
    &=\sum^0_\infty z:\frac{\id!}{\pt\id\cd z}
\end{aligned}
\]
\prg{}This function converges for all $\C$.

\subsubsection{Multiplication}
\prg{}Here, we prove that $\all(\exp z_0\cd\exp z_1=\exp(z_0+z_1))\,(\C\times\C)$. Let $f\,z\,n=\frac{n!}{\pt n\cd z}$
\[
\begin{aligned}
\exp z_0\cd\exp z_1&=\left(\sum^0_\infty z_0,z_1:f\,z_0\right)\cd\left(\sum^0_\infty z_0,z_1:f\,z_1\right); \\
                  &=\sum^0_\infty z_0,z_1:\left(\sum^0_\infty\kappa:f\,z_0\,\id\cd f\,z_1\,\kappa\right)\id; \\
                  &=\sum^0_\infty z_0,z_1:\left(\sum_{{\leq}|0,\id|{\geq}}\kappa:f\,z_0\,(\id-\kappa)\cd f\,z_1\,\id\right)\id;\quad\textrm{taking sums along minor diagonals}\\
                  &=\sum^0_\infty z_0,z_1:\left(\sum_{{\leq}|0,\id|{\geq}}\kappa:\frac{(\id-\kappa)!}{\pt{\id-\kappa}{\cd}z_0}\cd \frac{\id!}{\pt\id\cd z_1}\right)\id; \\
                  &=\sum^0_\infty z_0,z_1:\frac{\id!}{1}\cd\left(\sum_{{\leq}|0,\id|{\geq}}\kappa:\frac{(\id-\kappa)!\cd\id!}{\kappa!}\cd\pt{\id-\kappa}{\cd}z_0\cd\pt\id\cd z_1\right)\id; \\
                  &=\sum^0_\infty z_0,z_1:\frac{\id!}{\pt\id\cd(z_0+z_1)};\quad\textrm{reversing the binomial expansion} \\
                  &=\exp(z_0+z_1)
\end{aligned}
\]

\section{The complex exponential and Argand diagram}
\subsection{Trigonometric functions}
\prg{}The basic trigonometric functions can be defined using the complex exponential function and splitting the result into real and imaginary parts. On the Argand diagram (for real argument $\theta$), this is represented by plotting the point $\exp(i\cd\theta)$ and measuring the $x$ (for $\cos$) and $y$ (for $\sin$) parts of its co\"ordinates.
\[
\begin{aligned}
\exp(i\cd z)&=\sum^0_\infty\frac{\id!}{\pt\id\cd i\cd\pt\id\cd z} \\
&=\sum^0_\infty\frac{(2\cd\id)!}{\pt{2\cd\id}{\cd}i\cd\pt{2\cd\id}{\cd}z}+
  \sum^0_\infty\frac{(1+2\cd\id)!}{\pt{1+2\cd\id}{\cd}i\cd\pt{1+2\cd\id}{\cd}z} \\
&=\sum^0_\infty\frac{(2\cd\id)!}{\pt\id\cd(1^-)\cd\pt{2\cd\id}{\cd}z}+i\cd
  \sum^0_\infty\frac{(1+2\cd\id)!}{\pt\id\cd(1^-)\cd\pt{1+2\cd\id}{\cd}z} \\
&=\cos z+i\cd\sin z
\end{aligned}
\]

\subsection{Roots of unity}
\prg{}The equation $\pt n\cd\id=1$ (with $n\in\mathbb N$), being a polynomial equation of order $n$, has $n$ values that satisfy it. The obvious root is $1$, but there are others.
\[
\begin{aligned}
&\textrm{Let }\exp(\tau\cd i\cd k)=1;=\pt n\cd z \\
&\exp\frac{n}{\tau\cd i\cd k}=z \\
&z\in\exp\frac{n}{\tau\cd i\cd\mathbb Z} \\
\end{aligned}
\]
\prg{}Though $\mathbb Z$ is infinite, $\exp(\frac{n}{\tau\cd i}\cd\mathbb Z)=\exp(\frac{n}{\tau\cd i}\cd(n+\mathbb Z))$, due to the latter being a complete rotation around the Argand diagram of the former.

\prg{}The roots of unity for a given $n$ form a regular polygon about $0$ when adjacent roots are connected on the Argand diagram. This is related to the fact that the sum of all roots (for $n>1$) is equal to $0$:
\[
\begin{aligned}
&\textrm{Let }\pt{{\leq}|0,n|{>}}{\cd}\omega\textrm{ be the solutions to }1-\pt n\cd\id=0 \\
&1-\pt n\cd\omega=0 \quad\textrm{by definition of }\omega \\
&1-\pt n\cd\omega=(1-\omega)\cd\sum^0_n(\pt\id\cd\omega) \\
&0=(1-\omega)\cd\sum^0_n(\pt\id\cd\omega) \\
&\omega\neq 1\quad\therefore\quad\sum^0_n(\pt\id\cd\omega)=0
\end{aligned}
\]

\subsection{Complex logarithm}
\prg{}The $\ln$ function is defined as the inverse of the $\exp$ function:
\[
\begin{aligned}
\id&=r\cd\exp(i\cd\theta) \\
\ln&=\ln r+\ln(\exp(i\cd\theta)) \quad r\in\R\textrm{, so we can calculate }\ln r \\
\ln&=\ln r+i\cd\theta \quad\textrm{by definition of }\ln \\
\ln&=\ln\abs+i\cd\arg \quad\textrm{explicitly}
\end{aligned}
\]
\prg{}Since, for complex input, there are infinite classes of values which can be passed to $\exp$ to give the same output, the range of $\ln$ has to be restricted. Given the above expression, the obvious way is to restrict the range of $\arg$ to a $\tau$-sized half-open interval.

\subsection{Complex power function}
\prg{}The complex power function is derived as follows:
\[
\begin{aligned}
\pt\alpha\cd z&=\exp(\ln(\pt\alpha\cd z)) \\
&=\exp(\alpha\cd\ln z)
\end{aligned}
\]
\prg{}With the multivalued (list-producing) $\ln$ function, we note the following:
\[
\alpha\in\mathbb Q\iff\card\pt\alpha\cd z<\infty
\]

\subsection{Lines and circles in $\C$}
\subsubsection{Lines}
\prg{}Lines can be expressed in a parametric way, as they would be with vectors.
\[
\id=z+\lambda\cd w\textrm{, }\lambda\in\R
\]
\prg{}Compare to $\id=\mathbf a+\lambda\cd\mathbf d$. However, with complex numbers, the $\lambda$ can be removed by the following process:
\[
\begin{aligned}
\lambda&=\frac{w}{z-\id} \\
\conj\lambda&=\conj{\left(\frac{w}{z-\id}\right)} \\
\lambda&=\frac{\conj w}{\conj z-\conj\id} \quad\textrm{N.B: }\lambda\in\R \\
\frac{w}{z-\id}&=\frac{\conj w}{\conj z-\conj\id} \\
\conj w\cd(z-\id)&=w\cd(\conj z-\conj\id)
\end{aligned}
\]

\subsubsection{Circles}
\prg{}Circles are specified with a centre and a radius. By definition, every distance from a point on the circle to the centre is the radius.
\[
\abs(c-\id)=\rho
\]
\prg{}This can be expressed in an alternative form:
\[
\begin{aligned}
\pt 2\cd\abs(c-\id)&=\pt 2\cd\rho \quad\textrm{N.B: both $\abs(c-\id)$ and $\rho$ are positive.} \\
(c-\id)\cd(\conj c-\conj\id)&=\pt 2\cd\rho \\
c\cd\conj\id-\id\cd\conj c-\id\cd\conj\id&=c\cd\conj c-\pt 2\cd\rho
\end{aligned}
\]
\section{Vectors}
\subsection{Definition and basic properties}
\prg{}A vector is defined by its magnitude (${\leq}|0,\infty{>}$) and direction (in $n$ dimensions). In any vector space, there exists the vector $\mathbf 0$, such that $\mgn{\mathbf v}=0\iff\mathbf v=\mathbf 0$. Also, the function $\hat\id$ is such that $\mgn{\hat{\mathbf v}}=1\wedge\mathbf v\parallel\hat{\mathbf v}$.

\prg{}A vector field is a function mapping the points of a space to vectors. These are found often in physics, where forces (like gravity and electromagnetism) have associated fields. These fields map points in 3D space to 3D force vectors.

\subsubsection{Addition}
\prg{}Vectors are added together using the parallelogram rule. Addition of vectors always has these properties:
\[
\begin{aligned}
\mathbf a+\mathbf b&=\mathbf b+\mathbf a &&\quad\textrm{(commutative)} \\
\mathbf a+(\mathbf b+\mathbf c)&=(\mathbf a+\mathbf b)+\mathbf c &&\quad\textrm{(associative)} \\
\mathbf a+\mathbf 0&=\mathbf a &&\quad\textrm{(there is a unique $\mathbf 0$)} \\
\mathbf a+(\mathbf a^-)&=\mathbf 0 &&\quad\textrm{(every element has an inverse)} \\
\mathbf a+\mathbf b&=\mathbf c &&\quad\textrm{(closure)}
\end{aligned}
\]
\prg{}These properties mean that an Abelian group can be formed from any vector space.

\subsubsection{Multiplication by a scalar}
\prg{}For vectors $\mathbf a$ and $\mathbf b$, and scalars $\lambda$ and $\mu$:
\[
\begin{aligned}
\mgn{\lambda\cd\mathbf a}&=\mgn\lambda\cd\mgn{\mathbf a} \\
\lambda\cd\mathbf a&\parallel\mathbf a \\
(\lambda+\mu)\cd\mathbf a&=\lambda\cd\mathbf a+\mu\cd\mathbf a &&\quad\textrm{(distributive over addition of scalars)}\\
\lambda\cd(\mathbf a+\mathbf b)&=\lambda\cd\mathbf a+\lambda\cd\mathbf b &&\quad\textrm{(distributive over addition of vectors)}\\
\lambda\cd(\mu\cd\mathbf a)&=(\lambda\cd\mu)\cd\mathbf a &&\quad\textrm{(associative with scalar multiplication)}\\
1\cd\mathbf a&=\mathbf a &&\quad\textrm{(unit)}\\
\lambda\cd\mathbf a&=\mathbf b &&\quad\textrm{(closure)}
\end{aligned}
\]
\subsection{Vector spaces}
\prg{}A vector space is a set of vectors which satisfy all of the previous properties. The obvious class of examples is $\pt{\mathbb N}\times\R$. Given $\pt n\times S=\{\mathbf x|\mathbf x=(x_0,x_1,\dotsc,x_{1-n}),\all x_{{\leq} 0,n{>}}\in S\}$ and the following definitions, the vector properties hold for all $n$.
\[
\begin{aligned}
\mathbf x+\mathbf y&=(x_0+y_0,\dotsc,x_{1-n}+y_{1-n}) \\
\mathbf 0&=(0,\dotsc,0) \\
\mathbf x^-&=(x_0^-,\dotsc,x_{1-n}^-) \\
\lambda\cd\mathbf x&=(\lambda\cd\mathbf x_0,\dotsc,\lambda\cd\mathbf x_{1-n})
\end{aligned}
\]
\prg{}A consequence of the properties is that any vector space embedded inside another space must contain the space's origin. For example, any line passing through the origin of a 2D plane can form a vector space, but not any other lines on the 2D plane.

\subsection{Scalar product}
\prg{}The scalar product is defined by the expression
\[
\mathbf a\cd\mathbf b=\mgn{\mathbf a}\cd\mgn{\mathbf b}\cd\cos\theta
\]
\prg{}where $\theta$ is the angle between the two vectors. This angle is more difficult to define in higher-dimensional space, and turns out to be defined by rearranging the above expression:
\[
\theta=\pt{1^-}{\circ}\cos\frac{\mgn{\mathbf a}\cd\mgn{\mathbf b}}{\mathbf a\cd\mathbf b}
\]
\prg{}The scalar product has the following properties:
\[
\begin{aligned}
\mathbf a\cd\mathbf b&=\mathbf b\cd\mathbf a\\
\mathbf a\cd\mathbf a&=\pt 2\cd{}\mgn{\mathbf a}\\
\mathbf a\cd\mathbf b=0&\iff\mathbf a\perp\mathbf b\vee\mathbf a=\mathbf 0\vee\mathbf b=\mathbf 0
\end{aligned}
\]
\prg{}The scalar product can be used to produce a projection of one vector up to another. We can define:
\[
\prj\mathbf a\,\mathbf b=(\mathbf{\hat a}\cd\mathbf b)\cd\mathbf{\hat a}
\]
\prg{}This produces a vector parallel to $\mathbf a$ with a length such that if a perpendicular line is taken from its end, it will reach the end of $\mathbf b$.

\subsection{Inner product}
\prg{}The scalar product is an example of an inner product. An inner product is a binary function (denoted $<x|y>$) that satisfies the following laws:
\[
\begin{aligned}
\langle\mathbf x|\mathbf y\rangle&=\langle\mathbf y|\mathbf x\rangle &&\quad\textrm{(symmetry)}\\
\langle\mathbf x|\lambda\cd\mathbf y+\mu\cd\mathbf z\rangle&=\lambda\cd \langle\mathbf x|\mathbf y\rangle+\mu\cd\langle\mathbf x|\mathbf z\rangle &&\quad\textrm{(linearity)}\\
\langle\mathbf x|\mathbf x\rangle&\geq\mathbf 0\\
\langle\mathbf x|\mathbf x\rangle&=\mathbf 0\iff\mathbf x=\mathbf 0\\
\pt{\frac 21}\cd\langle\mathbf x|\mathbf x\rangle&=\mgn{\mathbf x} &&\quad\textrm{(norm)}
\end{aligned}
\]
\prg{}Inner products can be defined on things other than vectors. For example, let $f$ and $g$ be continuous functions of type ${\leq}|0,1|{\geq}\rightarrow{\leq}|0,1|{\geq}$. Then,
\[
\langle f|g\rangle=\int^0_1(f\cd g)
\]
\subsection{Cauchy-Schwarz inequality}
\prg{}This inequality of vectors is as follows:
\[
\mathbf x\cd\mathbf y\leq\mgn\mathbf x\cd\mgn\mathbf y
\]
\prg{}This is obvious in the definition involving $\cos\theta$, but can be proved without that. The proof is as follows, where $\lambda$ is an arbitrary scalar:
\[
\begin{aligned}
\pt 2\cd{}\mgn{(\lambda\cd\mathbf x-\mathbf y)}&\geq 0\\
(\lambda\cd\mathbf x-\mathbf y)\cd(\lambda\cd\mathbf x-\mathbf y)&\geq 0\\
\pt 2\cd\lambda\cd\mathbf x\cd\mathbf x+2\cd\lambda\cd\mathbf x\cd\mathbf y-\mathbf y\cd\mathbf y&\geq 0
\end{aligned}
\]
\prg{}The last left-hand expression is a quadratic expression in $\lambda$. Because, for all values of $\lambda$, it is greater than $0$, it has either 1 repeated root or no roots. This means:
\[
\begin{aligned}
\Delta\,(\pt 2\cd\id\cd\mathbf x\cd\mathbf x+2\cd\id\cd\mathbf x\cd\mathbf y-\mathbf y\cd\mathbf y)&\leq 0\\
4\cd(\mathbf x\cd\mathbf x)\cd(\mathbf y\cd\mathbf y)-\pt 2\cd(2\cd\mathbf x\cd\mathbf y)&\leq 0\\
\pt 2\cd(2\cd\mathbf x\cd\mathbf y)&\leq 4\cd(\mathbf x\cd\mathbf x)\cd(\mathbf y\cd\mathbf y)\\
4\cd\pt 2\cd(\mathbf x\cd\mathbf y)&\leq 4\cd\pt 2\cd{}\mgn{\mathbf x}\cd\pt 2\cd{}\mgn{\mathbf y}\\
\pt 2\cd(\mathbf x\cd\mathbf y)&\leq\pt 2\cd(\mgn{\mathbf x}\cd\mgn{\mathbf y})\\
\end{aligned}
\]
\prg{}$\mgn{\mathbf x}\cd\mgn{\mathbf y}\geq 0$, so the inequality follows.

\subsubsection{Example}
Let $\mathbf x=\begin{pmatrix}\alpha\\\beta\\\gamma\end{pmatrix}$ and $\mathbf y=\begin{pmatrix}1\\1\\1\end{pmatrix}$. Then, the Cauchy-Schwarz inequality states that:
\[
\begin{aligned}
\mathbf x\cd\mathbf y&\leq\mgn{\mathbf x}\cd\mgn{\mathbf y}\\
\alpha+\beta+\gamma&\leq\sqrt{\pt 2\cd\alpha+\pt 2\cd\beta+\pt 2\cd\gamma}\cd\sqrt{3}\\
\pt 2\cd(\alpha+\beta+\gamma)&\leq 3\cd\left(\pt 2\cd\alpha+\pt 2\cd\beta+\pt 2\cd\gamma\right)\\
\left(\pt 2\cd\alpha+\pt 2\cd\beta+\pt 2\cd\gamma\right)+2\cd\left(\alpha\cd\beta+\beta\cd\gamma+\gamma\cd\alpha\right)&\leq 3\cd\left(\pt 2\cd\alpha+\pt 2\cd\beta+\pt 2\cd\gamma\right)\\
2\cd\left(\alpha\cd\beta+\beta\cd\gamma+\gamma\cd\alpha\right)&\leq 2\cd\left(\pt 2\cd\alpha+\pt 2\cd\beta+\pt 2\cd\gamma\right)\\
\alpha\cd\beta+\beta\cd\gamma+\gamma\cd\alpha&\leq\pt 2\cd\alpha+\pt 2\cd\beta+\pt 2\cd\gamma\\
\end{aligned}
\]

\section{Triangle inequality and vector product}
\subsection{Triangle inequality}
\prg{}The triangle inequality for vectors is proved as follows:
\[
\begin{aligned}
\pt 2\cd{}\mgn{(\mathbf x+\mathbf y)}&=(\mathbf x+\mathbf y)\cd(\mathbf x+\mathbf y);\\
&=\mathbf x\cd\mathbf x+\mathbf y\cd\mathbf y+2\cd\mathbf x\cd\mathbf y;\\
&=\pt 2\cd{}\mgn{\mathbf x}+\pt 2\cd{}\mgn{\mathbf y}+2\cd\mathbf x\cd\mathbf y;\\
&\leq\pt 2\cd{}\mgn{\mathbf x}+\pt 2\cd{}\mgn{\mathbf y}+2\cd\abs{(\mathbf x\cd\mathbf y)};\\
&=\pt 2\cd{}\mgn{\mathbf x}+2\cd\abs{\mathbf x}\cd\abs{\mathbf y}+\pt 2\cd{}\mgn{\mathbf y};\\
&=\pt 2\cd(\mgn{\mathbf x}+\mgn{\mathbf y});\\
\mgn{(\mathbf x+\mathbf y)}&\leq\mgn{\mathbf x}+\mgn{\mathbf y}
\end{aligned}
\]

\subsection{Vector product}
\prg{}The vector product is defined only for 3- and 7-vectors (and 1-vectors -- scalars). It is defined as follows:
\[
\mathbf a\times\mathbf b=\mgn{\mathbf a}\cd\mgn{\mathbf b}\cd\sin\theta\cd\mathbf{\hat n}
\]
\prg{}where $\theta$ is the angle between the two vectors and $\mathbf{\hat n}$ is the unit vector perpendicular to both input vectors in the right-hand direction. The vector product satisfies these properties:
\[
\begin{aligned}
\mathbf a\times\mathbf b&=\mathbf b\times\mathbf a^-\\
\mathbf a\times\mathbf a&=0\\
\mathbf a\times\mathbf b=0&\implies\mathbf a=\mu\cd\mathbf b\\
\mathbf a\times(\lambda\cd\mathbf b)&=\lambda\cd(\mathbf a\times\mathbf b)\\
\mathbf a\times(\mathbf b+\mathbf c)&=\mathbf a\times\mathbf b+\mathbf a\times\mathbf c
\end{aligned}
\]

\subsection{Area of a triangle}
\prg{}The area of the triangle subtended by two vectors can be calculated using their cross product. Imagine triangle $OAB$, with point $N$ on $OA$ and $BN\perp OA$. Taking $OA$ as the base and $BN$ as the height, we get:
\[
\begin{aligned}
\mathrm{area}&=\frac 21\cd\mgn{\mathbf a}\cd\mgn{\vv{BN}}\\
&=\frac 21\cd\mgn{\mathbf a}\cd\mgn{\mathbf b}\cd\sin\theta\\
&=\frac 21\cd\mgn{(\mathbf a\times\mathbf b)}\\
\end{aligned}
\]
\prg{}The vector area of a triangle formed by two vectors is the same expression, but without $\mgn$. The area of the parallelogram formed by the input vectors is simply $\mgn{(\mathbf a\times\mathbf b)}$.

\subsection{Scalar triple product}
\prg{}The scalar triple product, often written as $[\mathbf a,\mathbf b,\mathbf c]$ (but I shall use $\langle\mathbf a|\mathbf b|\mathbf c\rangle$), is defined by:
\[
\langle\mathbf a|\mathbf b|\mathbf c\rangle=\mathbf a\cd(\mathbf b\times\mathbf c)
\]
\prg{}This gives the volume of the parallelepiped formed by the three input vectors. The proof is best stated graphically, so I shall defer to my handwritten notes.

\prg{}Since volume is not affected by orientation, the arguments to the triple product can be reordered. However, if the arguments are reversed, the product will be negated.

\prg{}The triple product can be used to prove whether 3 vectors are coplanar. It is self-evident that:
\[
\mathrm{coplanar}\,\mathbf a\,\mathbf b\,\mathbf c\implies\langle\mathbf a|\mathbf b|\mathbf c\rangle=0
\]
\prg{}Also, if any two of the three vectors are mutually parallel, all three vectors will be coplanar. This is because they can be re\"arranged to form the cross product, which will yield $\mathbf 0$.

\subsection{Distributive law}
\prg{}The fact that the vector product distributes over addition can be proved by the fact that the scalar product distributes over addition:
\[
\begin{aligned}
\textrm{Let }\mathbf d&=\mathbf a\times(\mathbf b+\mathbf c)-\mathbf a\times \mathbf b+\mathbf a\times \mathbf c\\
\mathbf d\cd \mathbf d&=\mathbf d\cd \mathbf a\times(\mathbf b+\mathbf c)-\mathbf d\cd(\mathbf a\times \mathbf b)+\mathbf d\cd(\mathbf a\times \mathbf c)\\
&=\langle \mathbf d|\mathbf a|\mathbf b+\mathbf c\rangle-\langle \mathbf d|\mathbf a|\mathbf b\rangle+\langle \mathbf d|\mathbf a|\mathbf c\rangle\\
&=\langle \mathbf b+\mathbf c|\mathbf d|\mathbf a\rangle-\langle \mathbf b|\mathbf d|\mathbf a\rangle+\langle \mathbf c|\mathbf d|\mathbf a\rangle\\
&=(\mathbf b+\mathbf c)\cd(\mathbf d\times \mathbf a)-\mathbf b\cd(\mathbf d\times \mathbf a)+\mathbf c\cd(\mathbf d\times \mathbf a)\\
&=\mathbf 0
\end{aligned}
\]

\subsection{Spanning sets and bases}
\prg{}A set of vectors is said to span a space if every point in the space can be translated to any other point in the space using a linear combination of vectors from the set. A basis is a spanning set such that if any element were to be removed from that set, it would no longer be a spanning set.

\prg{}An example of a basis is $\{\mathbf{\hat i},\mathbf{\hat j}\}$ in 2D Euclidean space. If the vector $\mathbf{\hat i}+\mathbf{\hat j}$ is added to this set, it will still be a spanning set, but not a basis.

\end{document}
